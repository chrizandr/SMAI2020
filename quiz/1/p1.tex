\begin{frame}
\section{}
Consider we are using PCA to compress face images using top K eigenvectors and then we do the reconstruction. Then
\begin{enumerate}
\item Compression (for face images) is lossy    % Ans
\item Compression (for face images) is lossless
\item Reconstruction will be bad for non-face images (say buildings)    % Ans
\item Reconstruction will be good for non-face images (say buildings)
\item None of these
\end{enumerate}
% Ans AC
\end{frame}

\begin{frame}
\section{}
Consider we are dong PCA to go from $R^2$ data to $R^1$. Consider each point is denoted by $(X_i,Y_i)$. Then in which of these situations will PCA work reasonably well:
\begin{enumerate}
\item $Y_i=X_i+10$    % Ans
\item $Y_i=X_i+10+\epsilon_i$ where $\epsilon_i\sim N(0,1)$   % Ans
\item $X_i^2+Y_i^2 = 10$
\item $X_i^2+Y_i^2 <= 10$
\item None of these
\end{enumerate}
% Ans: AB (A has linear,B has near-linear relationship which will get captured in PCA)
\end{frame}

\begin{frame}
\section{}
Consider we have data in $R^2$. Then the linear regression line and the PCA line
\begin{enumerate}
\item will always be the same
\item will never be the same
\item can sometimes be the same   % Ans
\item None of these
\end{enumerate}
% Ans: C
\end{frame}

\begin{frame}
\section{}
We want to do PCA using gradient descent. Then the update rule is
\begin{enumerate}
\item $u_{k+1} = \eta\Sigma u_k$
\item $u_{k+1} = (I+\eta\Sigma)u_k$   % Ans
\item $u_{k+1} = (I-\eta\Sigma)u_k$
\item None of these
\end{enumerate}
Assume that $\Sigma$ is the covariance matrix, $\eta$ is the learning rate.
% Ans: B (The +ve sign comes because we are solving a maximization problem)
\end{frame}

\begin{frame}
\section{}
PCA solves this problem:
\[\max_u u^T \Sigma u - \lambda (u^Tu-1)\]
where $\Sigma$ is the covariance matrix.
Which of the following are true regarding PCA
\begin{enumerate}
\item $\lambda$ is the variance captured by the eigen vector $u$    % Ans
\item Sum of variances captured by all eigenvectors is tr($\Sigma$)   % Ans
\item If all data points are on a line then at least one of the eigenvalues is 1
\item If all data points are on a line then at least one of the eigenvalues is 0    % Ans
\end{enumerate}
% Ans: ABD
\end{frame}

\begin{frame}
\section{}
Let $X=UDV^T$. Then
\begin{enumerate}
\item Columns of U are eigenvectors of $X^TX$
\item Columns of V a re eigenvectors of $X^TX$    % Ans
\item Rows of U are eigenvectors of $X^TX$
\item Rows of V are eigenvectors of $X^TX$
\item None of these
\end{enumerate}
% Ans: B
% $X^TX = VD^2V^T \implies X^TXV=VD^2 = V\Lambda$
\end{frame}

\begin{frame}
\section{}
Let $X=UDV^T$. Then
\begin{enumerate}
\item Columns of U are eigenvectors of $XX^T$   % Ans
\item Columns of V are eigenvectors of $XX^T$
\item Rows of U are eigenvectors of $XX^T$
\item Rows of V are eigenvectors of $XX^T$
\item None of these
\end{enumerate}
% Ans: A
% $XX^T = UD^2U^T \implies XX^TU=UD^2=U\Lambda$
\end{frame}

\begin{frame}
\section{}
Consider $X$ to be a square matrix of size $n\times n$ and $X=UDV^T$.
\begin{enumerate}
\item Both $X^TX$ and $XX^T$ have the same eigenvalues    % Ans
\item Both $X^TX$ and $XX^T$ have the same eigenvectors
\item $X$, $XX^TX$ and $XX^T$ have the same eigenvalues
\item $D^2$ contains the eigenvalues of $X^TX$ on its diagonal    % Ans
\item $D$ contains the eigenvalues of $X^TX$ on its diagonal
\item None of these
\end{enumerate}
% Ans: AD
\end{frame}

\begin{frame}
\section{}
Consider $X$ to be a square matrix of size $n\times n$ and $X=UDV^T$.
Then:
\begin{enumerate}
\item If rank(X) = $n$, D has all non-zero entries in diagonal.   % Ans
\item If rank(X) = $k$, D has $k$ zeros in diagonal
\item If rank(X) = $k$, D has $n-k$ zeros in diagonal   % Ans
\item if rank(X) = $n$ but $\vert A \vert $ is a very small number then, D takes the form $D=diag(d_1,d_2,..,\epsilon)$ where $\epsilon$ is a very small number   % Ans
\item None of these
\end{enumerate}
% Ans: ACD
\end{frame}

\begin{frame}
\section{}
Suppose you want to apply PCA to your data $X$
which is in 2D and you decompose $X$ as $UDV^T$. Then,
\begin{enumerate}
\item PCA can be useful if all elements of D are equal
\item PCA can be useful if all elements of D are not equal    % Ans
\item   $D$ is not full-rank if all points in $X$ lie on a straight line    % Ans
\item   $V$ is not full-rank if all points in $X$ lie on a straight line
\item   $D$ is not full-rank if all points in $X$ lie on a circle
\item None of these
\end{enumerate}
% Ans: BC
\end{frame}

\begin{frame}
\section{}
Given a set of 2D points $X$ on a line that makes 45 degree to the x-axis:
\[ X = \{ [1,1]^T, [2,2]^T, [3,3]^, [4,4]^T, [5,5]^T \}\]
We compute the covariance matrix, and its eigen values and eigen vectors. Then:

\begin{enumerate}
\item  $\lambda_2 = 0$    % Ans
\item  $\lambda_1 = \lambda_2$
\item  $\lambda_1 = -1$
\item  $\Sigma$ is singular   % Ans
\item none of the above
\end{enumerate}
% Ans: AD
\end{frame}

\begin{frame}
\section{}
Given a set of 2D points $X$ on a line that makes 45 degree to the x-axis:
\[ X = \{  [-2,2]^T, [-3,3]^, [-4,4]^T, [-5,5]^T [-6,6]^T \}\]
We compute the covariance matrix, and its eigen values and eigen vectors. Then:

\begin{enumerate}
\item  $\lambda_2 = 0$    % Ans
\item  $\lambda_1 = \lambda_2$
\item  $\lambda_1 = -1$
\item  $\Sigma$ is singular   % Ans
\item none of the above
\end{enumerate}
% Ans: AD
\end{frame}

\begin{frame}
\section{}
Given a set of 2D points $X$ on the vertical line $x_1=5$,
\[ X = \{ [5,1]^T, [5,2]^T, [5,3]^, [5,4]^T, [5,5]^T \}\]
We now add an additional
point $[4,3]^T$ to $X$.

We compute the covariance matrix, and its eigen values and eigen vectors. Then:


\begin{enumerate}
\item $\lambda_1 \geq \lambda_2$    % Ans
\item ${\bf u}_1$ and ${\bf u}_2$ are nearly orthogonal, but not perfectly orthogonal.
\item  $\Sigma$ is singular
\item  $\Sigma$ is diagonal   % Ans
\item None of the above.
\end{enumerate}
% Ans: AD
\end{frame}

\begin{frame}
\section{}
Given a set of 2D points $X$ on the vertical line $x_2=5$,
\[ X = \{ [1,5]^T, [2,5]^T, [3,5]^, [4,5]^T, [5,5]^T \}\]


We compute the covariance matrix, and its eigen values and eigen vectors. Then:


\begin{enumerate}
\item $\lambda_1 \geq \lambda_2$    % Ans
\item ${\bf \mu}$ is on the same line.    % Ans
\item  $\Sigma$ is singular   % Ans
\item  $\Sigma$ is diagonal   % Ans
\item None of the above.
\end{enumerate}
% Ans: ABCD
\end{frame}

\begin{frame}
\section{}
Set $X$ has 10 points. 5 of them are on a line that makes 45 degrees with the $x_1$ axis and another 5 from on a line that makes 135 degrees with the $x_1$ axis.

We compute the covariance matrix, and its eigen values and eigen vectors. Then:

\begin{enumerate}
\item $\lambda_1 = \lambda_2 \neq 0$
\item $\Sigma$ is singular
\item $\Sigma$ is diagonal
\item ${\bf \mu}$ is on either of these lines.
\item None of the above   % Ans
\end{enumerate}
% Ans: E
\end{frame}

\begin{frame}
\section{}
(use notations  and conventions from the class) Consider the problem of linear regression where we
minimize the loss
\[{\cal L}_1 = \frac{1}{N}\sum_{i=1}^N \alpha_i (y_i - {\bf w}^T{\bf x}_i)^2 + \lambda_1 g({\bf w})\] where $g()$ is a regularization term. We also write the loss in matrix form as \[ {\cal L}_2 = \frac{1}{N} [Y-{\bf X}{\bf w}]^TA[Y-{\bf X}{\bf w}] + \lambda_2 g({\bf w}). \]

\hrule
If ${\cal L}_1 = {\cal L}_2$ for all $\bf{w}$, then
\begin{enumerate}
\item $A$ is a diagonal matrix    % Ans
\item $A_{ij} = \alpha_i \cdot \alpha_j $
\item $A_{ii} = \alpha_i$ else zero   % Ans
\item $A_{ii} = \frac{1}{\alpha_i}$ else zero
\item none of the above
\end{enumerate}
% Ans: AC
\end{frame}

\begin{frame}
\section{}
(use notations  and conventions from the class) Consider the problem of linear regression where we
minimize the loss
\[{\cal L}_1 = \frac{1}{N}\sum_{i=1}^N \alpha_i (y_i - {\bf w}^T{\bf x}_i)^2 + \lambda_1 g({\bf w})\] where $g()$ is a regularization term. We also write the loss in matrix form as \[ {\cal L}_2 = \frac{1}{N} [Y-{\bf X}{\bf w}]^TA[Y-{\bf X}{\bf w}] + \lambda_2 g({\bf w}). \]

\hrule

If ${\bf A} = I$, $\alpha_i = 1$ for all $i$, and $\lambda_1 = \lambda_2 = 1$, then
\begin{enumerate}
\item Both the loss functions are identical i.e., ${\cal L}_1 = {\cal L}_2$   % Ans
\item The optima of the first objective ${\bf w}_1^*$ is same as the optima of ${\cal L}_2$, i.e., ${\bf w}_2^*$    % Ans
\item At the optima,  value of the losses are same. i.e., ${\cal L}_1^* = {\cal L}_2^*$   % Ans
\item ${\cal L}_1$ is a scalar and ${\cal L}_2$ is a vector
\item none of the above
\end{enumerate}
% Ans: ABC
\end{frame}

\begin{frame}
\section{}
(use notations  and conventions from the class) Consider the problem of linear regression where we
minimize the loss
\[{\cal L}_1 = \frac{1}{N}\sum_{i=1}^N \alpha_i (y_i - {\bf w}^T{\bf x}_i)^2 + \lambda_1 g({\bf w})\] where $g()$ is a regularization term. We also write the loss in matrix form as \[ {\cal L}_2 = \frac{1}{N} [Y-{\bf X}{\bf w}]^TA[Y-{\bf X}{\bf w}] + \lambda_2 g({\bf w}). \]

\hrule

If ${\bf A} = I$, $\alpha_i = 2$ for all $i$, and $\lambda_1 = \lambda_2 = 0$, then
\begin{enumerate}
\item Both the loss functions are identical i.e., ${\cal L}_1 = {\cal L}_2$
\item The optima of the first objective ${\bf w}_1^*$ is same as the optima of ${\cal L}_2$, i.e., ${\bf w}_2^*$    % Ans
\item At the optima, value of the losses are same. ${\cal L}_1^* = {\cal L}_2^*$
\item ${\cal L}_1$ is a scalar and ${\cal L}_2$ is a vector
\item none of the above
\end{enumerate}
% Ans: B
% L_1 = 2 * L_2 but the optimal weights remain same
\end{frame}

\begin{frame}
\section{}
(use notations  and conventions from the class) Consider the problem of linear regression where we
minimize the loss
\[{\cal L}_1 = \frac{1}{N}\sum_{i=1}^N \alpha_i (y_i - {\bf w}^T{\bf x}_i)^2 + \lambda_1 g({\bf w})\] where $g()$ is a regularization term. We also write the loss in matrix form as \[ {\cal L}_2 = \frac{1}{N} [Y-{\bf X}{\bf w}]^TA[Y-{\bf X}{\bf w}] + \lambda_2 g({\bf w}). \]

\hrule

If ${\bf A} = I$, $\alpha_i = 1$ for all $i$, and $\lambda_1 \neq \lambda_2 \neq 0$, then
\begin{enumerate}
\item The optimal parameters ${\bf w}^*$ is independent of $\lambda_i$.
\item The larger the lambda, the better the solution.
\item The smaller the lambda, the better the
solution
\item When lambda is nonzero (positive), loss will increase (since $g(w)$ is also positive in practice), better to use $\lambda=0$.
\item None of the above.    % Ans
\end{enumerate}
% Ans: E
% Ideally you would tune lambda to see what works best for given data
\end{frame}

\begin{frame}
\section{}
(use notations  and conventions from the class) Consider the problem of linear regression where we
minimize the loss
\[{\cal L}_1 = \frac{1}{N}\sum_{i=1}^N \alpha_i (y_i - {\bf w}^T{\bf x}_i)^2 + \lambda_1 g({\bf w})\] where $g()$ is a regularization term. We also write the loss in matrix form as \[ {\cal L}_2 = \frac{1}{N} [Y-{\bf X}{\bf w}]^TA[Y-{\bf X}{\bf w}] + \lambda_2 g({\bf w}). \]

\hrule
See ${\cal L}_2$ closely,
\begin{enumerate}
\item When $A$ is a diagonal matrix, this is equivalent to weighing each sample independently.    % Ans
\item When $A$ is not a  diagonal matrix, this loss does not make any sense. Don't use.
\item When $A$ is PD, we can do cholesky decomposition of $A$ as $LL^T$ and an equivalent formulation is possible in ${\cal L}_1$ is each sample getting transformed as ${\bf L}^T{\bf x}_i$ (as in LMNN/Metric Learning)   % Ans
\item When $A$ is a rank deficient matrix, an equivalent formulatiion is possible in ${\cal L}_1$ with a dimensionality reduction (this could be proved with eigen decomposition).    % Ans
\item None of the above
\end{enumerate}
% Ans: ACD
\end{frame}

\begin{frame}
\section{}
Consider a vocabulary of size $d$. One hot representation of a word $i$ is ``1'' at the location
(index) corresponding to that word and zero else where.

Given a document that contains $P$ words,  ${\bf w_1, \ldots, w_P}$, we compute
\[ {\bf x} = \sum_{i=1}^P {\bf w}_i \]
Then,
\begin{enumerate}
\item ${\bf x}$ is the histogram of the words, with $x_i$ as the frequency of $i$ th word.    % Ans
\item ${\bf x}$ is in $R^d$ independent of the number of words in the document.   % Ans
\item ${\bf x}$ is in $R^P$ independent of the vocabulary size.
\item $\sum_{i} {x}_i$ is $P$ ($x_i$ is the $i$ th element of ${\bf x}$)    % Ans
\end{enumerate}
%  Ans: ABD
\end{frame}

\begin{frame}
\section{}
Consider a document is represented by a histogram of the words in the document. ${\bf h}$ i.e., $h_i$ is the
number of occurrence of the $i$ th word in the document.

We define a linguistic operation: Paraphrasing (P1). P1 is defined as permuting sentences in a document and rewriting a sentence by permuting the words.

\begin{enumerate}
\item ${\bf h}$ is invariant to the P1    % Ans
\item ${\bf h}$ is not invariant to the P1
\item ${\bf h}$ is invariant under in which order the vocabulary is constructed (eg. "a to z" or "z to a"
\item a Euclidean distance computed over ${\bf h}_i$ and ${\bf h}_j$ is invariant under in which order the vocabulary is constructed (eg. "a to z" or "z to a".   % Ans
\end{enumerate}
% Ans: AD
\end{frame}

\begin{frame}
\section{}
Consider a document is represented by a histogram of the words in the document ${\bf h}$ i.e., $h_i$ is the
number of occurrence of the $i$ th word in the document.

We define a linguistic operation: Paraphrasing (P2). P2 is defined as replacing a set of words by their synonyms.

\begin{enumerate}
\item ${\bf h}$ is invariant to the P2
\item ${\bf h}$ is not invariant to the P2    % Ans
\item ${\bf h}$ is invariant under in which order the vocabulary is constructed (eg. "a to z" or "z to a"
\item a Euclidean distance computed over ${\bf h}_i$ and ${\bf h}_j$ is invariant under in which order the vocabulary is constructed (eg. "a to z" or "z to a"
\end{enumerate}
% Ans: B
\end{frame}

\begin{frame}
\section{}
A professor suspected that students while submiting home works are doing the paraphasing operations i.e., both P1 and P2. This resulted in failure of  some similarity tests.

Professor designs a $d\times d$  word similarity matrix ${\bf S}$ such that ${\bf S}_{ij} = {\bf S}_{ji} = 1$ if words $i$ and $j$ are synonyms and zero else. (Note: $d$ is the size of vocabulary).

Now to compare two documents, professor multiplies the histogram representations by ${\bf S}$.
\[ {\bf h}_i^\prime = {\bf S}{\bf h}_i  \](Note: ${\bf h}_i^\prime$ is the new representation. Also, note, after multiplying with the ${\bf S}$, the dimension does not change)

\begin{enumerate}
\item the new representation is invariant under the operation$P1$ and  $P2$. (i.e.,  All the plagiarism now will be detected.)  % Ans
\item the new representation is not invariant for $P2$ and it does not help.
\item the new representation helps for detecting people who have paraphrased with P2.  But now it
fails for the documents that were not paraphrased (like the original ones/sincere students!).
\item the idea is worth, but then ${\bf S}$ should not have made symmetric. with only one of ${\bf S}_{ij}$ or ${\bf S}_{ji}$ as 1. The method could have worked as expected.
\end{enumerate}
% Ans: A
% Diagonal elements of S are 1 since every word is a synonym of itself. Hence i^th element in new representation will be the frequency of i^th word + sum of the frequencies of all synonyms of i^th word
\end{frame}

\begin{frame}
\section{}
We want to compare two documents $i$ and $j$ which are represented as histogram (popular known as bag of words) of words $h_i$ and $h_j$.

Here is what four students argued:

\begin{enumerate}
\item histograms should be normalized by dividing by the number of words in the document so that the comparison operation becomes ``some what invariant'' to another linguistic operation: "summarization".   % Ans
\item Cosine distance is a popular distance to compare two documents using this  representation.    % Ans
\item we should remove the stop words (common words in the language) from the sentence so that the comparison will be more useful. Two documents have the same number of `the' does not mean any useful similarity between them.    % Ans
\end{enumerate}
% Ans: ABC
\end{frame}

\begin{frame}
\section{}
If $A=UDV^T$, then $A^TA$ is
\begin{enumerate}
\item $VD^2V^T$   % Ans
\item $UD^2U^T$
\item A square matrix   % Ans
\item is always full rank
\item none of the above
\end{enumerate}
% Ans: AC
\end{frame}

\begin{frame}
\section{}
Consider a matrix $A$ of size $m\times n$. Rank of $A$ is (choose one one most correct answer)
\begin{enumerate}
\item $\leq \min(m,n)$    % Ans
\item $\leq \max (m,n)$
\item $\geq \min(m,n)$
\item $\geq \max(m,n)$
\item $\frac{m+n}{2}$
\end{enumerate}
% Ans: A
\end{frame}

\begin{frame}
\section{}
A and B are two independent events such that $P(\overline A) = 0.4$ and $P(A \cap B) = 0.2$ Then $P(A \cap \overline B)$ is equal to
\begin{enumerate}
\item 0.4   % Ans
\item 0.2
\item 0.6
\item 0.8
\item None of  the above
\end{enumerate}
% Ans: A
\end{frame}

\begin{frame}
\section{}
If ${\bf A}$ is a $n\times n$ matrix, with every pair of columns orthogonal i.e., ${\bf a_i\cdot a_j = 0}\,\,\, \forall i,j$ and $||{\bf a_i}|| = 1$. Then:
\begin{enumerate}
\item ${\bf A^{-1}} = {\bf A}^T$.   % Ans
\item ${\bf A}{\bf A}^T = {\bf I}$    % Ans
\item ${\bf A}{\bf A}^T$ has only one 1 in every column and all others zero.
\item ${\bf A}^{-1}$ has only one 1 in every column and all others zero.
\item none of the above
\end{enumerate}
% Ans: AB
\end{frame}

\begin{frame}
\section{}
Product of Eigen values of a real square matrix is:
\begin{enumerate}
\item Determinant   % Ans
\item Rank
\item Trace
\item non-Negative
\item None of the above
\end{enumerate}
% Ans: A
\end{frame}

\begin{frame}
\section{}
$X\sim N(0,1)$, $Y\sim N(1,1)$ and $Z=X+Y$. Then,
\begin{enumerate}
\item $Z\sim N(0,2)$
\item $Z\sim N(0,1)$
\item $Z\sim N(1,1)$
\item $Z\sim N(1,2)$    % Ans
\item None of the above
\end{enumerate}
% Ans: D
\end{frame}
