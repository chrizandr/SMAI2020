\begin{frame}
\section{}
We know that solution to the problem of Maximize ${\bf w}^T{\bf A}{\bf w}$ subject to $||{\bf w}|| = 1$ is the eigen vector corresponding to the largest eigen value.

Note that we assume that a typical eigen value computation assumes to be returning (i) eigen values arranged in non-increasing order (ii) eigen vectors have unit L2 norm.

What is the solution to the problem of
Maximize ${\bf w}^T{\bf A}{\bf w}$ subject to $||{\bf w}||^2_2 = 2$

\begin{enumerate}
\item Eigen vector correspond to the second  eigen value.
\item Eigen vector correspond to the first eigen value.
\item 2*{\bf u} where ${\bf u}$ is the eigen vector correspond to the first eigen value.
\item $\sqrt{2}*{\bf u}$ where ${\bf u}$ is the eigen vector correspond to the first eigen value.   % Ans
\item None of the above.    % None
\end{enumerate}
% Ans: D
\end{frame}

\begin{frame}
\section{}
We are working with $N$ samples each of $d$ dimension.
Consider $N< d$

\begin{enumerate}
\item Solution to the problem of linear regression as a closed form can not be computed because the matrices are no longer compatble for multiplication.
\item Solution to the problem of linear regression as a closed form can not be computed because the matrix can not be inverted.   % Ans
\item Solution to the problem of ridge regularized linear regression as a closed form can not be computed because the matrix can not be inverted.
\item None of the above   % None
\end{enumerate}
% Ans: B (Covariance matrix becomes non full-rank, hence singular). But on adding lambda I, it can become full rank
\end{frame}

\begin{frame}
\section{}
We are working with $N$ samples each of $d$ dimension.
Consider $N\le d$

\begin{enumerate}
\item PCA can not be computed
\item While computing Eigen values, we will see $d$ zero eigen values.
\item While computing Eigen values, we will see at least $d-N$ zero eigen values.   % Ans
\item While computing Eigen values, we will see at max $d-N$ zero eigen values.   % Ans
\item We can not use eigen value/vector computation. We need to use SVD.    % Ans
\item None of the above.    % None
\end{enumerate}
% BCD
% Rank of covariance matrix will be N so that many non-zero eigen values. Rest of d-N eigenvalues are 0
\end{frame}

\begin{frame}
\section{}
We know the weighted Euclidean distance
\[ \tau = [{\bf x} - {\bf y}]^T[{\bf A}][{\bf x} - {\bf y}] \]

Where {\bf x} is a vector in $d$ dimension ${\bf A}$ is a square matrix

\begin{enumerate}
\item If ${\bf A}$ is a non-identity diagonal matrix. Then ``dist'' is a scaled version of Euclidean distance or ``$\tau  = \alpha$  Euclidean distance''   % Ans
\item If $k < d$ and $A = \sum_{i=1}^k {\bf u}_i{\bf u}_i^T$, with ${\bf u}_i$s as orthonormal. Then $A$ is rank deficient and Euclidean $\tau$ can not be computed.
\item If $k < d$ and $A = \sum_{i=1}^k {\bf u}_i{\bf u}_i^T$, with ${\bf u}_i$s as orthonormal. Then $\tau$ is equivalent to dimensioality reduction ${\bf x}^\prime = {\bf W}{\bf x}$ with ${\bf W}$ as $k\times d$ matrix with ${\bf u}_i^T$ as the $i$ th row.   % Ans
\item When ${\bf A}$ is non-diagonal matrix, $\tau$ can not be a metric.
\item None of the above.    % None
\end{enumerate}
% Ans: AC
\end{frame}

\begin{frame}
\section{}
Consider the covariance matrix $\Sigma$

\begin{enumerate}
\item $\Sigma$ is symmetric   % Ans
\item $\Sigma$ is PSD   % Ans
\item $\Sigma$ is Diagonal if the distribution is Normal.
\item $\Sigma$ can not be Diagonal if the distribution is Normal.
\item None of the above     % None
\end{enumerate}
% Ans: AB
\end{frame}
