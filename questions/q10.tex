\begin{frame}
\section{}
  Let $X=UDV^T$. Then
    \begin{enumerate}[label=(\Alph*)]
      \item Columns of U are eigenvectors of $X^TX$
          \item Columns of V are eigenvectors of $X^TX$   % Ans
          \item Rows of U are eigenvectors of $X^TX$
          \item Rows of V are eigenvectors of $X^TX$
          \item None of these   % None
    \end{enumerate}
\end{frame}

\begin{frame}
\section{}
  Let $X=UDV^T$. Then
     \begin{enumerate}[label=(\Alph*)]
       \item Columns of U are eigenvectors of $XX^T$    % Ans
      \item Columns of V are eigenvectors of $XX^T$
      \item Rows of U are eigenvectors of $XX^T$
      \item Rows of V are eigenvectors of $XX^T$
      \item None of these   % None
     \end{enumerate}
\end{frame}

\begin{frame}
\section{}
  Consider $X$ to be a square matrix of size $n\times n$ and $X=UDV^T$.
    \begin{enumerate}[label=(\Alph*)]
      \item Both $X^TX$ and $XX^T$ have the same eigenvalues    % Ans
          \item Both $X^TX$ and $XX^T$ have the same eigenvectors
          \item $X$, $XX^TX$ and $XX^T$ have the same eigenvalues
          \item $D^2$ contains the eigenvalues of $X^TX$ on its diagonal  % Ans
          \item $D$ contains the eigenvalues of $X^TX$ on its diagonal
          \item None of these   % None
    \end{enumerate}
\end{frame}

\begin{frame}
\section{}
  Consider $X$ to be a square matrix of size $n\times n$ and $X=UDV^T$.
    \begin{enumerate}[label=(\Alph*)]
      \item If rank(X) = $n$, D has all non-zero entries in diagonal. % Ans
     \item If rank(X) = $k$, D has $k$ zeros in diagonal
     \item If rank(X) = $k$, D has $n-k$ zeros in diagonal  % Ans
     \item if rank(X) = $n$ but $\vert A \vert $ is a very small number then, D takes the form  $D=diag(d_1,d_2,..,\epsilon)$ where $\epsilon$ is a very small number   % Ans
     \item None of these  % None
    \end{enumerate}
\end{frame}

\begin{frame}
\section{}
Suppose you want to apply PCA to your data $X$
  which is in 2D and you decompose $X$ as $UDV^T$. Then,
    \begin{enumerate}[label=(\Alph*)]
      \item PCA can be useful if all elements of D are equal
     \item PCA can be useful if all elements of D are not equal   % Ans
     \item   $D$ is not full-rank if all points in $X$ lie on a straight line   % Ans
     \item   $V$ is not full-rank if all points in $X$ lie on a straight line
     \item   $D$ is not full-rank if all points in $X$ lie on a circle
     \item None of these    % None
    \end{enumerate}
\end{frame}
