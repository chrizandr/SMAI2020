\begin{frame}
\section{}
  Consider the sigmoid function $g(z) = \frac{1}{1+ e^{-z}}$
    \begin{enumerate}[label=(\Alph*)]
      \item when $z=0$, $g(z) = 0.5$    % Ans
          \item when $z$ is negative, $g(z)$ is also negative.
          \item $g(z)$ is always in the range of $[0,1]$    % Ans
          \item $g(z)$ is always in the range of $[-1,1]$
    \end{enumerate}
\end{frame}

\begin{frame}
\section{}
Consider the sigmoid function $g(\alpha, z) = \frac{1}{1+ e^{-\alpha z}}$
  where $\alpha$ is a positive real number.
     \begin{enumerate}[label=(\Alph*)]
       \item if $\alpha_1 > \alpha_2$, then $g(\alpha_1, z) \geq g(\alpha_2, z)$ for all z
      \item if $\alpha_1 > \alpha_2$, then $g(\alpha_1, z) \leq g(\alpha_2, z)$ for all z
      \item if $\alpha_1 > \alpha_2$, then $g(\alpha_1, z) \geq g(\alpha_2, z)$ for all z in the range $[-1,1]$
      \item if $\alpha_1 > \alpha_2$, then $g(\alpha_1, z) \geq g(\alpha_2, z)$ for all z in the range $[1,2]$    % Ans
      \item if $\alpha_1 > \alpha_2$, then $g(\alpha_1, z) \geq g(\alpha_2, z)$ for all z in the range $[-2,-1]$
     \end{enumerate}
\end{frame}

\begin{frame}
\section{}
Consider the sigmoid function $g(z) = \frac{1}{1+ e^{-z}}$
   Then $g^\prime(z)$ i.e., derivative of $g(z)$ with respect to z
    \begin{enumerate}[label=(\Alph*)]
      \item is always positive for all values of $z$    % Ans
      \item is constant, i.e., derivative is independent of $z$.
      \item $\frac{1}{1+e^z}$
      \item $\frac{e^{-z}}{(1+e^{-z})^2}$   % Ans
      \item $g(z)(1-g(z))$    % Ans
    \end{enumerate}
\end{frame}

\begin{frame}
\section{}
Consider the sigmoid function $g(z) = \frac{1}{1+ e^{-z}}$
   Then $1-g(z)$ is
    \begin{enumerate}[label=(\Alph*)]
      \item is in the range of $[0,1]$.   % Ans
      \item $\frac{1}{1+e^z}$
      \item $\frac{e^{-z}}{1+e^{-z}}$   % Ans
      \item is in the range of $[-1,0]$.
      \item is in the range of $[-1,+1]$.   % Ans
    \end{enumerate}
\end{frame}

\begin{frame}
\section{}
  You know the popular sigmoid function $g(z) = \frac{1}{1+ e^{-z}}$, and also the $\tanh(z) = \frac{e^{z} - e^{-z}}{e^{z}+e^{-z}}$
     \begin{enumerate}[label=(\Alph*)]
       \item $tanh(z)$ is in the range of $[0,1]$
      \item $tanh(z)$ is in the range of $[-1,+1]$    % Ans
      \item $tanh(z) = 2g(2z) - 1$    % Ans
      \item when $z=0$, $tanh(z)$ is $0$.   % Ans
      \item when $z=0$, $tanh(z)$ is $0.5$.
     \end{enumerate}
\end{frame}
