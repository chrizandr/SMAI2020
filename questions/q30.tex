\begin{frame}
\section{}
Among the following problems, a clustering algorithm is mostly appropriate for:

\begin{enumerate}[label=(\Alph*)]
\item Predicting the rainfall in HYD in 2022.
\item For a robot to decide the direction to travel to Himalaya 105 class room.
\item To detect Credit Card transactions that are Frauds    % Ans
\item To translate a sentence from Hindo to Telugu
\item All the above  % None
\end{enumerate}
\end{frame}

\begin{frame}
\section{}
Consider a Divisive Clustering Algorithm with two steps:
\begin{itemize}
\item Create an MST
\item Successively remove the longest or largest edges.
\end{itemize}
Assume there are 100 samples, and all edges are of unique length. If we have removed 5 edges, the number of clusters is:
\begin{enumerate}[label=(\Alph*)]
\item 5
\item $2^5$
\item $5^2$
\item 6   % Ans
\item None of the above.    % None
\end{enumerate}
\end{frame}

\begin{frame}
\section{}
Consider a Divisive Clustering Algorithm with two steps:
\begin{itemize}
\item Create an MST
\item Successively remove the longest or largest edges.
\end{itemize}
Assume there are 100 samples, and all edges are of unique length.


\begin{enumerate}[label=(\Alph*)]
\item This algorithm is yielding a globally optimal solution to a specific objective.    % Ans
\item Every run of this algorithm can give different solution and therefore, this is sensitive to the ordering/indices of the samples in the set.
\item Since this is a global optima, there can not exist a better clustering algorithm.
\item Since there are better/other clustering algorithms, the final solution is only locally optimal.
\item The objective function that this algorithm minimizes is the following: \[ \sum_{l} \sum_{x_i, x_j \in C_l} d(x_i, x_j)^2 \]    % Ans
\end{enumerate}
\end{frame}


\begin{frame}
\section{}
Assume there are $N$ samples in a data set, the number of distinct ways in which we can cluster this set is:

\begin{enumerate}[label=(\Alph*)]
\item $N$
\item $2^N$
\item $N!$
\item $_NC_2$
\item None of the above  % None`    % Ans
\end{enumerate}
\end{frame}

\begin{frame}
\section{}
Consider a Divisive Clustering Algorithm with two steps:
\begin{itemize}
\item Create an MST
\item Successively remove the longest or largest edges.
\end{itemize}
Assume there are 100 samples, and all edges are unique length.

What can be a {\bf bad} termination criteria?

\begin{enumerate}[label=(\Alph*)]
\item Stop when all the left out edges are less than $p$
\item Stop when there are no more edges to remove.    % Ans
\item Stop when the length of the next largest is less than half of the edge removed in the previous step?
\item Stop when the length of the next largest is more than half of the edge removed in the previous step?    % Ans
\item Average length of leftout edges is more than average length of removed edges.   % Ans
\end{enumerate}
\end{frame}
