\begin{frame}
\section{}
See the Sec 3.3 (equations)

Why is $a\in R^+$ written there?

\begin{enumerate}[label=(\Alph*)]
\item It could have been $a\in R^-$
\item $a$ negative does not lead to $K$ being PSD   % Ans
\item It is a typo.
\end{enumerate}
% Desc You might have read the notes on Kernels and SVMs at: https://www.dropbox.com/s/qryziuo3u143q5e/KERNEL-REVIEW.pdf?dl=0
\end{frame}

\begin{frame}
\section{}
See the Sec 3.3. Assume
$\alpha_i \in R^+$; $\beta_i \in R^+$; $\kappa_i({\bf p},{\bf q})$ being a valid kernel. Also $K$ and $L$ are some positive integers.

Then a new kernel $\kappa(\cdot,\cdot) = $
\begin{enumerate}[label=(\Alph*)]
\item $\sum_{i=1}^K \kappa_i ({\bf p},{\bf q})$ is a valid kernel.    % Ans
\item $\prod_{i=1}^L \kappa_i ({\bf p},{\bf q})$ is a valid kernel.   % Ans
\item $\sum_{i=1}^K \alpha_i \kappa_i ({\bf p},{\bf q})$ is a valid kernel.   % Ans
\item $\prod_{i=1}^L \beta_i \kappa_i ({\bf p},{\bf q})$ is a valid kernel.   % Ans
\item $\sum_{i=1}^K \alpha_i \kappa_i ({\bf p},{\bf q})$ + $\prod_{i=1}^L\beta_i \kappa_i ({\bf p},{\bf q})$ is a valid kernel    % Ans
\end{enumerate}
% Desc You might have read the notes on Kernels and SVMs at: https://www.dropbox.com/s/qryziuo3u143q5e/KERNEL-REVIEW.pdf?dl=0
\end{frame}

\begin{frame}
\section{}
See the pseudo-code for Kernel Perceptron (Algorithm 3). Assime the kernel to be
$({\bf x}^T{\bf y})^2$
\begin{enumerate}[label=(\Alph*)]
\item The initialization $\alpha_i=0$ is a must. With no other initialization, this algorithm will not work (say will not converge)
\item Step of computing Kernel Matrix (step 2) should have been in side the loop (repeat structure).
\item Since this is now Kernelized, with any data (irrespective of whether the data is linearly separable or not), this algorithm will converge.
\item For data that is linearly separable, this algorithm will give you a linear decision boundary.
\item None of the above.  % None    % Ans
\end{enumerate}
% Desc You might have read the notes on Kernels and SVMs at: https://www.dropbox.com/s/qryziuo3u143q5e/KERNEL-REVIEW.pdf?dl=0
\end{frame}


\begin{frame}
\section{}
Look at the equation (94) related to the objective function:
\begin{enumerate}[label=(\Alph*)]
\item This is an L1 softmargin SVM  % Ans
\item This is an L2 softmargin SVM
\item There is a typo. $\xi_i$ should be replaced as $\xi_i^2$
\item There is a typo. LHS will have to be $j({\bf w}{\bf \xi})$, since $\xi$ is another variable that we need to optimize.
\item None of the above.    % None
\end{enumerate}
% Desc You might have read the notes on Kernels and SVMs at: https://www.dropbox.com/s/qryziuo3u143q5e/KERNEL-REVIEW.pdf?dl=0
\end{frame}

\begin{frame}
\section{}
Consider the decision making rule. ``one side of a line (in 2D) is +ve class and other side of a line is -ve class''  Figure 7 shows that VC dimension of a class of functions (lines) in 2D is 3.

What is the VC dimension in 1D for a function class. If $x>\theta$, positive, else negative.

Write your answer in the space provided.

(Sample answer (possibly incorrect): 1 )
% FIB
\end{frame}
